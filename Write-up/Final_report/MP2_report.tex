\documentclass[10pt,a4paper]{report}
\usepackage{algorithm}
\usepackage{algorithmicx}
\usepackage{algpseudocode}
\usepackage{amsfonts}
\usepackage{amsmath}
\usepackage{amssymb}
\usepackage{bbm}
\usepackage{fullpage}
\usepackage{graphicx}
\usepackage[numbers]{natbib}
\usepackage[utf8]{inputenc}
\author{Nathan Cunningham}
\title{Application of sequential Monte Carlo methods for the clustering of multiple datasets}
\begin{document}

\maketitle

\begin{abstract}
In this paper I 
\end{abstract}
\section{Introduction}

Bit on cluster analysis, bit on multiple datasets, applications of this to e.g. gene expression data.

A little bit on sequential monte carlo methods.






\section{Literature}
In their paper \cite{kirk2012bayesian}, \citeauthor{kirk2012bayesian} propose an unsupervised method for the integration of multiple datasets. Their work is applicable to a number of data types: gaussian, gaussian processes, time series data, multinomial data.
MDI paper summary \cite{kirk2012bayesian}

SMC paper summary \cite{griffin2014sequential}

Maybe bit on Sarah Wade's paper?
\cite{wade2015bayesian}
\section{Methods}
The algorithm presented here is a combination of the work by \citeauthor{griffin2014sequential} and \citeauthor{kirk2012bayesian}
\begin{algorithm}
\caption{Gibbs sampler}
 \begin{algorithmic}[1]
  \State Initialise $\Gamma$ matrix of prior allocation weights and $\Phi$ matrix of dataset concordance values
  \For{i = 1, \dots, number of iterations}
  \State Conditional on $\Gamma_{i-1}$ and $\Phi_{i-1}$ update the cluster labels, $c_{i}$, using alg. \ref{alg:pf}
  \State Conditional on $c_{i}$ update $\Gamma_i$ and $\Phi_i$
  \EndFor
\end{algorithmic}
\end{algorithm}


\begin{algorithm}
\caption{Particle filter to update cluster allocations}
\label{alg:pf}
 \begin{algorithmic}[1]
  \For{i = 1, \dots, n} \Comment{Loop over observations}
  \For{m = 1, \dots, M} \Comment{Loop over particles}
  \For{j = 1, \dots, d} \Comment{Loop over datasets}
  \State Sample $c^{(m)}_{i, j}$ \Comment{Propose a cluster for each datum}
  \State $q(c^{(m)}_{i,j} = k) \propto k^*(y_{i,j}|c_{i,j}^{(m)} = k) \gamma_{i, k, j}$ 
  \State $\xi^{(m)} =  \xi^{(m)} \times \gamma_{i, k, j}(1+\phi_{i})k^*(y_{i,k}|c_{i,j}^{(m)} = k)$ 
  \EndFor
  \EndFor
  \State Resample particles according to $\xi^{(m)}$
  \EndFor
  \State Update cluster labels using allocation in particle with largest $\xi^{(m)}$

\end{algorithmic}
\end{algorithm}

Where
\begin{equation}
\label{eq:likelihood}
k^*(y_{i, k}|c_{i, k}^{(m)} = k) = (\mathbf{y_{i, k}} - \mathbf{\mu_k}) \mathbf{\Sigma^{-1}} (\mathbf{y_{i, k}} - \mathbf{\mu_k})^\top
\end{equation}
\begin{equation}
\label{eq:phi}
\Phi \text{ is a measure of cluster label correspondence across datasets}
\end{equation}
\begin{equation}
\label{eq:gamma}
\gamma_{i, k, j} \text{ is a prior weight for assigning observation i, in dataset k to cluster j}
\end{equation}
\section{Example application}
Comparison of this versus independent clustering of the datasets

Use on multinomial data and gaussian data


\section{Conclusions and proposals for future work}
Some success. Improvements over independent clustering...?

Future work needed: Updating of concordance values by particle? Outputting of more than one particle?

Parallelisation of the code to speed things up.

Feature selection in cluster analysis

Application to real-life data (genomics England)





\bibliographystyle{plainnat}
\bibliography{bibliography.bib}
\end{document}